\documentclass[a4paper, 12pt]{article}
\usepackage{xcolor, enumerate, geometry, times, CJK}
\usepackage[backend=biber,style=apa]{biblatex}
\usepackage[colorlinks=true, linkcolor=blue]{hyperref}
\usepackage{bookmark}
\addbibresource{article.bib}
\geometry{margin=2cm}
\setlength{\parindent}{2em}

\title{COVID-19 is natural or artificial generation\\
\small
B076060052 姚燕珍 \ B081020055 戴潔伶\\
B073040031 葉星佑 \ B073040047 楊志璿
}
\date{}

\begin{document}
\begin{CJK*}{UTF8}{bsmi}
    \maketitle

    A news which is translated by \textcite{202009150112} states that "Yan Li-Meng
    provided evidence that the COVID-19 is artificial and easy to make."
    This report's initial claim is that the COVID-19 was artificially recombinant 
    and released from a Chinese laboratory. And we found two premises support this
    initial claim. First, COVID-19's biological characteristics were inconsistent 
    with a naturally zoonotic virus \textcite{202009150112}. Because 
    \textcite{yan_li_meng_2020_4028830} supplied evidence that states the template
    of COVID-19 should be a laboratory product synthesized by two bat coronaviruses 
    ZC45 and ZXC2. Additionally, \textcite{202008140064} also supplies evidence by
    stating the laboratory can synthesize ZC45 and ZXC2 and is capable created 
    COVID-19 this kind of coronaviruses. 
    Second, it claims that COVID-19 is was released from China. \textcite{20211119}
    supports this premise by supplying evidence that states the index case is a 
    seafood vendor at the South China Seafood Market and all-cause in the early 
    stages have a relationship with the market which means South China Seafood Market
    spread COVID-19. Moreover, \textcite{202008140064} supplied evidence by indicating
    ZC45 and ZXC21 coronaviruses, which could be the backbone of Covid-19, were found
    by the Chinese Military Research Institute and showed in NIH hence database. 
    It can be concluded from this information that the COVID-19 is created from the 
    laboratory artificially. Which is supported by two premises: (1) The virus was
    reformed from bat's coronavirus and (2) COVID-19 was released from China.\\

    There is another report that belief COVID-19 is not leaking from laboratories, 
    but some animals \parencite{202111190104}. The counterclaim of this article is
    COVID-19 is natural generation. One premise of the counterclaim is that COVID-19
    variated from animal virus, which was started by Michael Worobey, Professor 
    of Ecology and Evolutionary Biology \parencite{52133480, s41586-020-2012-7}.
    To support this premise, \textcite{202111190104} supplied evidence stated by 
    Professor Worobey that most of the early COVID-19 confirmed cases happened in
    the Hua-Nan market, specifically cases that were related to stalls selling 
    raccoon dogs. Moreover, \textcite{52133480} supplied evidence by describing 
    a study published in the authoritative journal Nature that the COVID-19 virus 
    naturally occurs in bats as there is 96\% similarity between bat virus, RaTG13,
    and COVID-19 virus. 
    Another premise that supports the counterclaim is \textcite{202111190104} stated
    that origin of the virus from a lab was ridiculous. Professor Worobey said that 
    if the virus came from the laboratory, the early cases of infecting COVID-19 should
    be around the lab rather than in faraway markets. Additionally, a publication 
    from WHO claims that the release of the virus due to laboratory accidents is 
    an "extremely unlikely way" \parencite{57861634, who}.
    It can be concluded from this information that the coronavirus could not have 
    leaked from a laboratory because COVID-19 shared many similarities with nature 
    animal viruses and only a low probability that COVID-19 leaking from China's 
    laboratory.\\

    We posit that the information presented in the initial claim appears to be
    creditable because it was provided by an authoritative news agency, \textcite{202009150112}, 
    was based on the research by Dr. Li-Meng \textcite{yan_li_meng_2020_4028830}, 
    an international and professional virologist. Taiwan's vice-president \textcite{24227}
    praised CNA, a historical and authoritative agency, and all the evidence provided
    stemmed from a paper by Yan. Yan, who has published over 2000 articles, is
    an international and professional virologist and she stated that the COVID-19
    virus is manufactured and sourced from a China laboratory \parencite{3535570}.
    The purpose of the report in the initial claim does not appear to have any 
    malicious intent, but it did threaten to China's reputation \parencite{202008140064}.
    The information of the counterclaim by \textcite{202111190104} appears to be
    insufficiently persuasive to prove that the COVID virus mutated from an animal virus.
    CAN \parencite{202111190104}, an historical and authoritative agency, provide 
    evidence to support this counterclaim, but this evidence does not appear to be
    as clear as the evidence supplying the initial claim. WHO published information that
    \textcite{who} status there is a low probability of viruses leaking from laboratories.
    The information presented in this piece has no purpose as it merely presents long-term
    survey research.\\

    In conclusion, it is our opinion that the initial claim is flip-information because 
    it does not have any malicious intent and just describes the result of Yan's paper.
    The main purpose of the initial claim is to present the virologist's research and 
    the journalist has no intention to harm anyone. Yan's paper as described in 
    the initial claim was only to explain the origin of COVID-19 and did not appear to 
    have any intention of slandering anyone. There is a lot of research about the epidemic 
    that stated COVID-19 could be released from the laboratory which supports the initial
    claim. According to a convention submitted by the Australian government, the Chinese
    government admitted there is a high risk of human-made viruses leaking from the 
    laboratories \parencite{202106280087}. During NCKU's public speech, Professor Lai
    Ming-Zhao disclosed that scientists had added parts of the sequence to the COVID-19
    virus \parencite{5829288}.

    \printbibliography[title=REFERENCES]

\end{CJK*}
\end{document}