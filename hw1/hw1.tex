%!TEX program = xelatex
\documentclass[a4paper, 12pt]{article}

\usepackage{xeCJK, times, xcolor, enumerate, geometry, fontspec}
\usepackage[backend=biber, style=apa]{biblatex}
\usepackage[xetex, colorlinks=true, linkcolor=blue, citecolor=black]{hyperref}
\usepackage{bookmark}

\addbibresource{hw1.bib}
\geometry{margin=2cm}
\setCJKmainfont[BoldFont=教育部標準楷書]{教育部標準楷書}
\setromanfont{Times New Roman}

\begin{document}
\begin{enumerate}[I.]
    \item []{\color{blue}初始主張}
    \item [Q.] 本調查將關注哪些新聞報導(或少於 5 分鐘的影片)?提供引用和參考文獻.
    \item 呱吉爆料龍龍原是《吐槽大會》的成員卻因不滿群組無人理會她的留言而退出 (\cite{2091301})
          \begin{enumerate}[A.]
              \item [Q.] 本文或影片中提出的{\color{blue}初始主張}是什麼?
              \item 龍龍在群組內發言無人理會
                    \begin{enumerate}[1.]
                        \item[Q.] 支持{\color{blue}初始主張}的{\underline{一個前提}}是什麼?
                        \item 龍龍在群組有發言
                              \begin{enumerate}[a.]
                                  \item [Q.] 什麼{\underline{證據}}支持此{\underline{前提}}?
                                  \item 龍龍提及:「X好啊我講話都沒人理啊。」(\cite{2091700})
                                  \item 呱吉:「講了一些東西,但是沒有人回她」(\cite{eFs8bNzDAHg7200s})
                              \end{enumerate}
                        \item [Q.] 支持{\color{blue}初始主張}的{\underline{另一個前提}}是什麼?
                        \item 龍龍覺得不被尊重
                              \begin{enumerate}[a.]
                                  \item [Q.] 什麼{\underline{證據}}支持此{\underline{前提}}?
                                  \item 龍龍提及:「但有些人只會人身攻擊\&挖人隱私後就退出了」(\cite{2778844612438695})
                                  \item 龍龍提及:「為什麼大家可以一直造謠還覺得沒有問題呢?」(\cite{2091700})
                              \end{enumerate}
                        \item [Q.] 從這些資訊中可以得出什麼{\underline{結論}}?
                        \item 因龍龍認為不被尊重,所以退群
                    \end{enumerate}
          \end{enumerate}
    \item [] {\color{red}對立主張}
    \item [Q.] 哪些新聞報導、影片或多方資訊來源對初始主張提出對立主張(也就是反對意見)?
    \item 本次調查將圍繞《龍龍親揭退群真相》一文展開 (\cite{20211001003526-260404})
          \begin{enumerate}[A.]
              \item [Q.] 本文或影片中提出的{\color{red}對立主張}是什麼?
              \item 龍龍退出《火烤呱吉》是因為檔期延期與原本有工作相撞
                    \begin{enumerate}[1.]
                        \item[Q.] 支持{\color{red}對立主張}的{\underline{一個前提}}是什麼?
                        \item 《火烤呱吉》檔期延期
                              \begin{enumerate}[a.]
                                  \item [Q.] 什麼{\underline{證據}}支持此{\underline{前提}}?
                                  \item 龍龍臉書發文提及:「退出《火烤呱吉》我講了一百次是因為檔期延期」(\cite{2091700,froggyroast})
                                  \item 原於 4/10 週五 晚間8點 西門紅樓劇場 演出,但實延至 8/14 (\cite{10158175426336462,froggyroast})
                              \end{enumerate}
                        \item [Q.] 支持{\color{red}對立主張}的{\underline{另一個前提}}是什麼?
                        \item 與原本有工作時間衝突
                              \begin{enumerate}[a.]
                                  \item [Q.] 什麼{\underline{證據}}支持此{\underline{前提}}?
                                  \item 龍龍臉書發文提及:「退出《火烤呱吉》我講了一百次是因為檔期延期」(\cite{2091700,froggyroast})
                              \end{enumerate}
                        \item [Q.] 從這些資訊中可以得出什麼{\underline{結論}}?
                        \item 活動有從 4/10 延期至 8/14,但是否與龍龍工作相撞所以辭演不知。
                    \end{enumerate}
          \end{enumerate}
    \item [] % Empty Line
    \item [] % Empty Line
    \item [] 討論
    \item [Q.] 對於{\color{blue}初始主張}中所提供之資訊的可信度,你持何種立場?
    \item 皆為口述或發文,實情未知。
          \begin{enumerate}[A.]
              \item [Q.]提出{\color{blue}初始主張}的來源(例如個人或組織)是否在所檢視的領域或學科中具有學術知識或經驗資格?
              \item 是,為當事人。
              \item [Q.]還有誰可為{\color{blue}初始主張}提供之資訊的可信度背書?
              \item 無,當事人自述。
              \item [Q.]{\color{blue}初始主張}中的訊息/資訊是否存在有潛在的傷害意圖?
              \item 無,此為當事人主觀感受。
          \end{enumerate}
    \item []
    \item [Q.] 對於{\color{red}對立主張}中所提供之資訊的可信度,你持何種立場?
    \item 是否與龍龍工作相撞所以辭演不知。
          \begin{enumerate}[A.]
              \item [Q.] 提出{\color{red}對立主張}的來源(例如個人或組織)是否在所檢視的領域或學科中具有學術知識或經驗資格?
              \item 是,為當事人。
              \item [Q.] 還有誰可為{\color{red}對立主張}提供之資訊的可信度背書?
              \item 卡米地臉書活動分享。(\cite{10158175426336462,froggyroast})
              \item [Q.] {\color{red}對立主張}中的訊息/資訊是否存在有潛在的傷害意圖?
              \item 無
          \end{enumerate}
    \item [] % Empty Line
    \item [] 結論
    \item [Q.] 從你的角度看,{\color{blue}初始主張}中所提供的哪些具體資訊是造假訊息、無惡意的傷害訊息、惡意訊息、或錯誤訊息?
    \item 無從判斷是否造假或是否惡意。個人認為無造假、無惡意也無錯誤訊息。
          \begin{enumerate}[A.]
              \item [Q.] 為什麼{\color{blue}初始主張}中的該處資訊是造假訊息、惡意訊息、或錯誤訊息?
              \item 不存在。
                    \begin{enumerate}[1.]
                        \item [Q.] 哪些其他來源的額外證據支持此結論?
                        \item 呱吉:「我自己也曾經講了一兩句很好笑的事情,結果都沒有人回。」(\cite{eFs8bNzDAHg7227s})
                    \end{enumerate}
              \item [Q.] 還有什麼原因讓你認為{\color{blue}初始主張}中的該處訊息是造假訊息、惡意訊息、或錯誤訊息?
              \item 無,此為當事人主觀感受。
                    \begin{enumerate}[1.]
                        \item [Q.] 哪些其他來源的額外證據支持此結論?
                        \item 無。
                    \end{enumerate}
          \end{enumerate}
\end{enumerate}

\printbibliography[title=參考文獻]

\end{document}