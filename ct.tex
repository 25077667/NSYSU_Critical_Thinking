%!TEX program = xelatex
\documentclass[a4paper, 12pt]{article}

\usepackage{xeCJK, times, xcolor, enumerate, geometry, fontspec}
\usepackage[backend=biber,style=numeric,sorting=none]{biblatex}
\usepackage[xetex, colorlinks=true, linkcolor=blue]{hyperref}
\usepackage{bookmark}

\bibliography{ct.bib}
\geometry{margin=2cm}
\setlength{\parindent}{2em}
\setCJKmainfont[BoldFont=教育部標準楷書]{教育部標準楷書}
\setmainfont{Times New Roman}

\begin{document}
    \begin{enumerate}[I.]
        \item[]{\color{blue}初始主張}
        \item[Q.] 本調查將關注哪些新聞報導(或少於 5 分鐘的影片)?提供引用和參考文獻.
        \item{\color{gray}一句話簡答} % TODO: {一句話簡答}
            \begin{enumerate}[A.]
                \item [Q.]本文或影片中提出的{\color{blue}初始主張}是什麼?
                \item {\color{gray}一句話簡答} % TODO: {一句話簡答}
                \begin{enumerate}[1.]
                    \item[Q.] 支持{\color{blue}初始主張}的{\underline{一個前提}}是什麼?
                    \item{\color{gray}一句話簡答} % TODO: {一句話簡答}
                    \begin{enumerate}[a.]
                        \item [Q.] 什麼{\underline{證據}}支持此{\underline{前提}}?
                        \item {\color{gray}一句話簡答} % TODO: {一句話簡答}
                        \item {\color{gray}一句話簡答} % TODO: {一句話簡答}
                    \end{enumerate}
                    \item [Q.]支持{\color{blue}初始主張}的{\underline{另一個前提}}是什麼?
                    \item{\color{gray}一句話簡答} % TODO: {一句話簡答}
                    \begin{enumerate}[a.]
                        \item [Q.] 什麼{\underline{證據}}支持此{\underline{前提}}?
                        \item {\color{gray}一句話簡答} % TODO: {一句話簡答}
                        \item {\color{gray}一句話簡答} % TODO: {一句話簡答}
                    \end{enumerate}
                    \item [Q.]從這些信息中可以得出什麼{\underline{結論}}?
                    \item {\color{gray}一句話簡答} % TODO: {一句話簡答}
                \end{enumerate}
            \end{enumerate}
        \item [] {\color{red}對立主張}
        \item[Q.] 哪些新聞報導、影片或多方信息來源對初始主張提出對立主張(也就是反對意見)?
        \item {\color{gray}一句話簡答} % TODO: {一句話簡答}
        \begin{enumerate}[A.]
            \item [Q.]支持{\color{red}對立主張}的{\underline{前提}} 是什麼?
            \item {\color{gray}一句話簡答} % TODO: {一句話簡答}
                \begin{enumerate}[1.]
                    \item [Q.] 什麼{\underline{證據}}支持此{\underline{前提}}?
                    \item {\color{gray}一句話簡答} % TODO: {一句話簡答}
                    \item {\color{gray}一句話簡答} % TODO: {一句話簡答}
                \end{enumerate}
                \item [Q.]支持{\color{red}對立主張}的{\underline{另一個前提}}是什麼?
                \item{\color{gray}一句話簡答} % TODO: {一句話簡答}
                \begin{enumerate}[1.]
                    \item [Q.] 什麼{\underline{證據}}支持此{\underline{前提}}?
                    \item {\color{gray}一句話簡答} % TODO: {一句話簡答}
                    \item {\color{gray}一句話簡答} % TODO: {一句話簡答}
                \end{enumerate}
                \item [Q.]從這些信息中可以得出什麼{\underline{結論}}?
                \item {\color{gray}一句話簡答} % TODO: {一句話簡答}
        \end{enumerate}

        \item [] % Empty Line
        \item [] 討論
        \item [Q.] 對於{\color{blue}初始主張}中所提供之信息的可信度,你持何種立場?
        \item {\color{gray}一句話簡答} % TODO: {一句話簡答}
            \begin{enumerate}[A.]
                \item [Q.]提出{\color{blue}初始主張}的來源(例如個人或組織)是否在所檢視的領域貨學科中具有學術知識或經驗資格?
                \item {\color{gray}一句話簡答} % TODO: {一句話簡答}
                \item [Q.]還有誰可為{\color{blue}初始主張}提供之信息的可信度背書?
                \item {\color{gray}一句話簡答}
                \item [Q.]{\color{blue}初始主張}中的訊息/信息是否存在有潛在的傷害意圖?
                \item {\color{gray}一句話簡答}
            \end{enumerate}
        \item []
        \item [Q.] 對於{\color{red}對立主張}中所提供之信息的可信度,你持何種立場?
        \item {\color{gray}一句話簡答} % TODO: {一句話簡答}
            \begin{enumerate}[A.]
                \item [Q.]提出{\color{red}對立主張}的來源(例如個人或組織)是否在所檢視的領域貨學科中具有學術知識或經驗資格?
                \item {\color{gray}一句話簡答} % TODO: {一句話簡答}
                \item [Q.]還有誰可為{\color{red}對立主張}提供之信息的可信度背書?
                \item {\color{gray}一句話簡答}
                \item [Q.]{\color{red}對立主張}中的訊息/信息是否存在有潛在的傷害意圖?
                \item {\color{gray}一句話簡答}
            \end{enumerate}
        \item [] % Empty Line
        \item [] 結論
        \item [Q.] 從你的角度看,{\color{blue}初始主張}中所提供的哪些具體信息是造假訊息、無惡意的傷害訊息、惡意訊息、或錯誤訊息?
        \item {\color{gray}一句話簡答} % TODO: {一句話簡答}
            \begin{enumerate}[A.]
                \item [Q.]為什麼{\color{blue}初始主張}中的該處信息是造假訊息、惡意訊息、或錯誤訊息?
                \item {\color{gray}一句話簡答} % TODO: {一句話簡答}
                    \begin{enumerate}[1.]
                        \item [Q.]哪些其他來源的額外證據支持此結論?
                        \item {\color{gray}一句話簡答} % TODO: {一句話簡答}
                    \end{enumerate}
                \item [Q.]還有什麼原因讓你認為{\color{blue}初始主張}中的該處訊息是造假訊息、惡意訊息、或錯誤訊息?
                \item {\color{gray}一句話簡答} % TODO: {一句話簡答}
                    \begin{enumerate}[1.]
                        \item [Q.]哪些其他來源的額外證據支持此結論?
                        \item {\color{gray}一句話簡答} % TODO: {一句話簡答}
                    \end{enumerate}
            \end{enumerate}
    \end{enumerate}

    \printbibliography{參考文獻}




\end{document}