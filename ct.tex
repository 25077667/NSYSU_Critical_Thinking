\documentclass[a4paper, 12pt]{article}
\usepackage{amsmath, amsfonts, CJKutf8, times, color, xcolor, enumerate, geometry}
\usepackage[backend=biber,style=numeric,sorting=none]{biblatex}
\usepackage[colorlinks=true, linkcolor=blue]{hyperref}
\usepackage{bookmark}


\bibliography{ct.bib}
\geometry{margin=2cm}
\setlength{\parindent}{2em}


\begin{document}
\begin{CJK}{UTF8}{bkai}
    \begin{enumerate}[I]
        \item[]{\color{blue}初始主張}
        \item[Q.] 本調查將關注哪些新聞報導(或少於 5 分鐘的影片)?提供引用和參考文獻.
        \item 一句話簡答 % TODO: {一句話簡答}
              \begin{enumerate}
                  \item [Q.] 本文或影中提出的{\color{blue}初始主張}是什麼?
                  \item 一句話簡答 % TODO: {一句話簡答}
                        \begin{enumerate}
                            \item[Q.] 支持{\color{blue}初始主張}的{\underline{一個前提}}是什麼?
                            \item
                        \end{enumerate}
              \end{enumerate}
        \item [] {\color{red}對立主張}
        \item[Q.] 哪些新聞報導、影片或多方信息來源對初始主張提出對立主張(也就是反對意見)?
        \item 一句話簡答 % TODO: {一句話簡答}
        \item [] % Empty Line
        \item [] 討論
        \item [Q.] 對於{\color{blue}初始主張}中所提供之信息的可信度,你持何種立場?
        \item 一句話簡答 % TODO: {一句話簡答}
        \item []
        \item [Q.] 對於{\color{red}對立主張}中所提供之信息的可信度,你持何種立場?
        \item 一句話簡答 % TODO: {一句話簡答}
        \item [] % Empty Line
        \item [] 結論
        \item [Q.] 從你的講度來看,中的哪些具體信息是造假訊息、無惡意的傷害訊息、惡意訊息、或錯誤訊息?
        \item 一句話簡答 % TODO: {一句話簡答}
    \end{enumerate}

    \printbibliography{參考文獻}



\end{CJK}
\end{document}