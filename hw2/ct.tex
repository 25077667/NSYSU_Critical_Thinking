\documentclass[a4paper, 12pt]{article}
\usepackage{xcolor, enumerate, geometry, times, CJK}
\usepackage[backend=biber,style=apa]{biblatex}
\usepackage[colorlinks=true, linkcolor=blue]{hyperref}
\usepackage{bookmark}
\addbibresource{ct.bib}
\geometry{margin=2cm}
\setlength{\parindent}{2em}

\title{Electromagnetic wave may hurt the human body\\
\small
B076060052 姚燕珍 \ B081020055 戴潔伶\\
B073040031 葉星佑 \ B073040047 楊志璿
}
\date{}

\begin{document}
\begin{CJK*}{UTF8}{bsmi}
      \maketitle
      \begin{enumerate}[I.]
            \item[] {\color{blue}INITIAL CLAIM}
            \item[Q.] {What news report (or video which is less than 5 minutes) will this investigation focus on? (
                  Provide citation and reference support.)}
            \item Mobile phone electromagnetic waves may incur physical and mental health risks \parencite{AIA970701}. % TODO: {\color{gray} One declarative sentence answer}
                  \begin{enumerate}[A.]
                        \item [Q.] What is the {\color{blue}initial claim} presented in this article or video?
                        \item Electromagnetic waves that cell phones produce and use increase the chance of a person developing
                              cancer in their body.
                              \begin{enumerate}[1.]
                                    \item[Q.] What is \underline{one premise} that supports the {\color{blue}initial claim}?
                                    \item Electromagnetic waves from mobile phones can affect the human body.
                                          \begin{enumerate}[a.]
                                                \item [Q.] What \underline{evidence} supports the \underline{premise}?
                                                \item In 2006, the World Health Organization released a new
                                                      disease called "Electromagnetic Hypersensitivity" that may
                                                      affect the human central nervous system, immune system,
                                                      cardiovascular, reproductive system, visual system \parencite{36069}.
                                                \item According to the report of Yan of National Yi-Lan University,
                                                      the participant who uses a mobile phone for a long time will
                                                      experience more symptoms of physical discomfort,
                                                      including dizziness, weariness, and headache \parencite{R9631008}.
                                          \end{enumerate}
                                    \item [Q.] What is \underline{another premise} that supports the {\color{blue}initial claim}?
                                    \item Electromagnetic waves can increase cancer risk.
                                          \begin{enumerate}[a.]
                                                \item [Q.] What \underline{evidence} supports the \underline{premise}?
                                                \item Israeli scientists pointed out that people who use mobile
                                                      phones every day for several hours have a 50\% higher chance
                                                      of developing parotid gland cancer than
                                                      those who don't use mobile phones at all \parencite{epaper9706,10.1093/aje/kwm325}.
                                                \item According to a book called “Electromagnetic Waves and
                                                      Human Health”, it mentioned there is an association
                                                      between electromagnetic waves and childhood leukemia
                                                      which is cancer of the body's blood-forming issues \parencite{16094}.
                                          \end{enumerate}
                                    \item [Q.] What can be \underline{concluded} from this information?
                                    \item According to the information, we find that electromagnetic wave
                                          from cell phones produces could affect different body systems and
                                          increase the risk of cancer. % TODO: {\color{gray} One declarative sentence answer}
                              \end{enumerate}
                  \end{enumerate}
            \item [] {\color{red}COUNTERCLAIM}
            \item [Q.] What news report, video or multiple information sources present a {\color{red} counterclaim} (i.e., objection)?
            \item Are electromagnetic waves terrible? The base station is out of communication, and the primitives are disturbed. \parencite{20210822}
                  \begin{enumerate}[A.]
                        \item [Q.] What is the {\color{red} counterclaim} presented in this article or video?
                        \item Electromagnetic waves that cell phones produce and use does not increase the chance of a person developing cancer in their body.
                              \begin{enumerate}[1.]
                                    \item[Q.] What is \underline{one premise} that supports the {\color{red}counterclaim}?
                                    \item Its energy is not enough to affect humans.
                                          \begin{enumerate}[a.]
                                                \item [Q.] What \underline{evidence} supports the \underline{premise}?
                                                \item The visible spectrum is $10^5$ or higher energy than cell phones electromagnetic waves.
                                                \item Einstein says: $E=h\nu$ in his paper "On the Electrodynamics of Moving Bodies". \parencite{Einstein}
                                          \end{enumerate}
                                    \item [Q.] What is \underline{another premise} that supports the {\color{red}counterclaim}?
                                    \item The relationship between electromagnetic waves and cancer has never been established.
                                          \begin{enumerate}[a.]
                                                \item [Q.] What \underline{evidence} supports the \underline{premise}?
                                                \item The poster says: "It is Class 2B carcinogen". \parencite{mp109151,iarc}
                                                \item Epidemiological evidence is limited, and animal experimental evidence is lacking. \parencite{nhri}
                                          \end{enumerate}
                                    \item [Q.] What can be \underline{concluded} from this information?
                                    \item It is not enough to affect humans, but do not reject it has a relationship between electromagnetic waves and cancer.
                              \end{enumerate}
                  \end{enumerate}

            \item [] % Empty Line
            \item [] DISCUSSION
            \item [Q.] What is your position toward the credibility of the information presented in the {\color{blue}initial claim}?
            \item We thought this report's has low credibility. This information is provided by the Qi-Xin clinic, however,
                  there is no strong academically supports.
                  \begin{enumerate}[A.]
                        \item [Q.] Is the source (e.g., person or organization) making the {\color{blue}initial claim} academically knowledgeable
                              or experientially qualified in the field or subject area under examination?
                        \item No, the Qi-Xin clinic is just a health checking center, but they don't have the cancer experiential.
                        \item [Q.] Who else endorses the credibility of the information presented in the {\color{blue}initial claim}?
                        \item The Cellular Phone Use and Risk of Benign and Malignant Parotid Gland Tumors—A Nationwide Case-Control Study \parencite{10.1093/aje/kwm325}.
                        \item [Q.] Is there an underlying intention to harm in the message/information in the {\color{blue}initial claim}?
                        \item No, they didn't want have any intention to harm any people. They just remind people to take care of it.
                  \end{enumerate}
            \item []
            \item [Q.] What is your position toward the credibility of the information presented in the {\color{red} counterclaim}?
            \item We thought the counterclaim has much credibilities to show that cell phones produce and use does
                  not increase the chance of a person developing cancer.
                  \begin{enumerate}[A.]
                        \item [Q.] Is the source (e.g., person or organization) making the {\color{red}counterclaim} academically knowledgeable
                              or experientially qualified in the field or subject area under examination?
                        \item The author of this report \parencite{20210822} is a medical professor in UCSF and he published over 200 papers of medical
                              researches.
                        \item [Q.] Who else endorses the credibility of the information presented in the {\color{red}counterclaim}?
                        \item The famous physicist, Einstein, shows the formula of energy and electromagnetic wave relationship.
                        \item [Q.] Is there an underlying intention to harm in the message/information in the {\color{red}counterclaim}?
                        \item Prof Lin. verified the myth of cell phones' electromagnetic wave, and he didn't want to harm people.
                  \end{enumerate}
            \item [] % Empty Line
            \item [] CONCLUSION
            \item [Q.] From your perspective, what specific information presented in the {\color{blue}initial claim} or {\color{red}counterclaim}
                  is {\color{blue} mis}-, {\color{blue}flip}-, {\color{blue}dis}-, or {\color{blue}mal}-information?
            \item The {\color{blue}initial claim} is {\color{blue} mis}-information.
                  \begin{enumerate}[A.]
                        \item [Q.] Why is the information {\color{blue} mis}-, {\color{blue}flip}-, {\color{blue}dis}-, or {\color{blue}mal}-information?
                        \item The initial claim didn't want to harm any people.
                              \begin{enumerate}[1.]
                                    \item [Q.] What evidence from other source(s) supports this conclusion?
                                    \item The report says that it is necessary to adjust the habit of using mobile
                                          phones to reduce the physical and mental health damage caused by mobile phones.
                                          Everyone in the mobile phone family should be concerned \parencite{AIA970701}.
                              \end{enumerate}
                        \item [Q.] Why else do you believe the information is {\color{blue} mis}-, {\color{blue}flip}-, {\color{blue}dis}-, or {\color{blue}mal}-information?
                        \item There is no any academical supporting this initial claim.
                              \begin{enumerate}[1.]
                                    \item [Q.] What evidence from other source(s) supports this conclusion?
                                    \item The definition of the WHO's 2B carcinogen is that
                                          epidemiological evidence is limited, and animal experimental evidence is lacking \parencite{nhri}.
                              \end{enumerate}
                  \end{enumerate}
      \end{enumerate}

      \printbibliography[title=REFERENCES]

\end{CJK*}
\end{document}