\documentclass[a4paper, 12pt]{article}
\usepackage{xcolor, enumerate, geometry, times, CJK}
\usepackage[backend=biber,style=apa]{biblatex}
\usepackage[colorlinks=true, linkcolor=blue]{hyperref}
\usepackage{bookmark}
\addbibresource{article.bib}
\geometry{margin=2cm}
\setlength{\parindent}{2em}

\title{Electromagnetic wave may hurt the human body\\
\small
B076060052 姚燕珍 \ B081020055 戴潔伶\\
B073040031 葉星佑 \ B073040047 楊志璿
}
\date{}

\begin{document}
\begin{CJK*}{UTF8}{bsmi}
    \maketitle

    There is a report that said that mobile phones
    bring civilization but may incur physical and mental health
    risks, which is published at \parencite{AIA970701}. This
    report's initial claim is that the electromagnetic waves
    that cell phones produce and use increase the chance of a
    person developing cancer in their body. We found two premises
    support this initial claim. First, it is a report that cites
    World Health Organization (WHO) disease list. It mentions
    a disease, which is called "Electromagnetic Hypersensitivity",
    which may affect the human central nervous system, immune
    system, cardiovascular, reproductive system, visual
    system \parencite{36069} . Also, there is a report was published
    by National Ilan University. The participant who uses a mobile
    phone for a long time will experience more symptoms of
    physical discomfort, including dizziness, weariness, and
    headache \parencite{R9631008} .
    Second, Israeli scientists
    pointed out that people who use mobile phones every day
    for long hours have a 50\% higher chance of developing
    parotid gland cancer than people who don't use mobile phones
    \parencite{10.1093/aje/kwm325}. And according to a book
    called "Electromagnetic Waves and Human Health", it
    mentioned there is an association between electromagnetic
    waves and childhood leukemia which is a cancer of the body's
    blood-forming tissues \parencite{16094} .
    It can be concluded from this information that cell
    phones' electromagnetic waves might harm people's health.
    \\

    There is another report that belief electromagnetic wave from
    calling phone won't hurt the human body non increase any
    risk of disease \parencite{20210822} . The counterclaim of
    this article is electromagnetic waves that cell phones
    produce and use do not increase the chance of a person
    developing cancer in their body. One premise of the counterclaim
    is that the radiation of electromagnetic waves is weak
    and not enough to affect the human body. Evidence that supports
    this premise is the visible spectrum is $10^5$ or higher
    energy than cell phones electromagnetic waves
    \parencite{20210822} . Also, Einstein, a famous physicist,
    mentioned a formula that is $E=h\nu$ which can calculate
    the strength of the electromagnetic wave of a cell
    phone that is very weak and can't hurt the human body
    \parencite{Einstein} .
    Another premise that supports the
    counterclaim is the relationship between electromagnetic
    waves of cell phones and cancer has never been established.
    Dr. Wu, who works in the Department of Hematology and Oncology
    in Taipei United Hospital, the risk of cell phone cancer are
    very low, and its harm cannot even be compared with smoking
    and obesity \parencite{mp109151} . And According to the
    report of WHO, this potentially harmful scientific
    evidence is weak that even at high frequencies (above 1800MHz)
    electromagnetic fields will not increase the risk of
    cancer, non-even low-frequency cell phone electromagnetic
    waves \parencite{nhri} .
    It can be concluded that the radiation of electromagnetic
    waves that cell phones produce is weaker than the sunlight
    and cancer is not necessarily related to cell phone
    electromagnetic waves.\\

    We posit that the information presented in the initial claim
    is questionable because the clinic doesn't have any
    experts in electromagnetic waves or cancer. Ching-Shun
    clinic is a physical examination center, but they don't have
    an oncologist that all their expertise and certification are
    only about health check service, so we believe the people
    who forward this initial claim have enough academically
    knowledgeable or are experientially qualified. However,
    \parencite{10.1093/aje/kwm325} published the report "The
    Cellular Phone Use and Risk of Benign and Malignant Parotid
    Gland Tumors" in which that information supports
    the initial claim.
    There appears to be no malicious intent in the initial claim
    about the main purpose of the report because it reminds people
    to pay more attention to the electromagnetic waves coming from
    their mobile phones.\\

    In the counterclaim, it is more credible that the author is a 
    well-known expert in the medical field. \parencite{20210822} who
    is the author of the counterclaim, is a medical professor
    in UCSF and he published over 200 papers of medical researches.
    This could also be supported by Einstein theory that the
    information about cell phone electromagnetic waves can’t hurt the
    human body \parencite{Einstein} . This counterclaim does not
    have an underlying intention because it just wants to reduce
    people’s misunderstandings about the electromagnetic waves
    of mobile phones.\\

    In conclusion, it appears that the initial claim is
    mis-information. One reason why we believe that this information
    is mis-information is the main purpose of the initial claim
    is positive and there is no malicious intent. Another reason
    why we believe that this information is mis-information is that
    there are not enough experimental results to support it have
    the relationship between electromagnetic waves of cell phones
    and cancer.

    \printbibliography[title=REFERENCES]

\end{CJK*}
\end{document}